\documentclass[12pt]{article}
 
\usepackage[margin=1in]{geometry} 
\usepackage{amsmath,amsthm,amssymb}
\usepackage{hyperref}
\usepackage{graphicx}
\usepackage{xcolor}
\usepackage[many]{tcolorbox}
\tcbuselibrary{listings}
\usepackage{listings}

\definecolor{lg}{HTML}{f0f0f0}

\newtcblisting{pycode}{
    colback=lg,
    boxrule=0pt,
    arc=0pt,
    outer arc=0pt,
    top=0pt,
    bottom=0pt,
    colframe=white,
    listing only,
    left=15.5pt,
    enhanced,
    listing options={
        basicstyle=\small\ttfamily,
        keywordstyle=\color{blue},
        language=Python,
        showstringspaces=false,
        tabsize=2,
        numbers=left,
        breaklines=true
    },
    overlay={
        \fill[gray!30]
        ([xshift=-3pt]frame.south west)
        rectangle
        ([xshift=11.5pt]frame.north west);
    }
}

\lstset{
    language=Python,
    basicstyle=\small\ttfamily,
}

 
\begin{document}
 
\title{Home Assignment 1}
\author{Cristian Manuel Abrante Dorta - 888248\\
CS-E4650 - Methods of data mining}

\maketitle
\section{Task 1}
To cite works, put them in the template.bib file and use~\cite{sutton2018reinforcement}.

\section{Task 2}
To embed code snippets in the report, you can use the \texttt{pycode} environment.

\begin{pycode}
def main():
    print("Hello World")
\end{pycode}

\section{Question 1}

If you add a figure, you can refer to it using Figure.~\ref*{fig:fig1}.

\begin{figure}[h] 
	\centering  % Remember to centre the figure
    \includegraphics[width=0.3\columnwidth]{img/training.pdf}
	\caption{This is a sample figure.}
	\label{fig:fig1}
\end{figure}

\bibliographystyle{ieeetr}
\bibliography{bibliography}  % Modify template with your bibliography name
\end{document}
